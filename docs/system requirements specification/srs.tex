%Copyright 2014 Jean-Philippe Eisenbarth
%This program is free software: you can 
%redistribute it and/or modify it under the terms of the GNU General Public 
%License as published by the Free Software Foundation, either version 3 of the 
%License, or (at your option) any later version.
%This program is distributed in the hope that it will be useful,but WITHOUT ANY 
%WARRANTY; without even the implied warranty of MERCHANTABILITY or FITNESS FOR A 
%PARTICULAR PURPOSE. See the GNU General Public License for more details.
%You should have received a copy of the GNU General Public License along with 
%this program.  If not, see <http://www.gnu.org/licenses/>.

%Based on the code of Yiannis Lazarides
%http://tex.stackexchange.com/questions/42602/software-requirements-specification-with-latex
%http://tex.stackexchange.com/users/963/yiannis-lazarides
%Also based on the template of Karl E. Wiegers
%http://www.se.rit.edu/~emad/teaching/slides/srs_template_sep14.pdf
%http://karlwiegers.com
\documentclass{scrreprt}
\usepackage{listings}
\usepackage{underscore}
\usepackage[bookmarks=true]{hyperref}
\usepackage[utf8]{inputenc}
\usepackage[english]{babel}
\hypersetup{
    bookmarks=false,    % show bookmarks bar?
    pdftitle={Software Requirement Specification},    % title
    pdfauthor={Jean-Philippe Eisenbarth},                     % author
    pdfsubject={TeX and LaTeX},                        % subject of the document
    pdfkeywords={TeX, LaTeX, graphics, images}, % list of keywords
    colorlinks=true,       % false: boxed links; true: colored links
    linkcolor=blue,       % color of internal links
    citecolor=black,       % color of links to bibliography
    filecolor=black,        % color of file links
    urlcolor=purple,        % color of external links
    linktoc=page            % only page is linked
}%
\def\myversion{1.0 }
\date{}
%\title
\usepackage{hyperref}
\begin{document}

\begin{flushright}
    \rule{16cm}{5pt}\vskip1cm
    \begin{bfseries}
        \Huge{SOFTWARE REQUIREMENTS\\ SPECIFICATION}\\
        \vspace{1.9cm}
        for\\
        \vspace{1.9cm}
        Simulated Computing System\\
        \vspace{1.9cm}
        \LARGE{Version \myversion}\\
        \vspace{1.9cm}
        Prepared by António Pedro Fraga\\
        \vspace{1.9cm}
        Cranfield University\\
        \vspace{1.9cm}
        \today\\
    \end{bfseries}
\end{flushright}

\tableofcontents


\chapter*{Revision History}

\begin{center}
    \begin{tabular}{|c|c|c|c|}
        \hline
	    Name & Date & Reason For Changes & Version\\
        \hline
	    21 & 22 & 23 & 24\\
        \hline
	    31 & 32 & 33 & 34\\
        \hline
    \end{tabular}
\end{center}

\chapter{Introduction}

\section{Purpose}
This document is a Software Requirements Specification of a project developed under under two modules, \textbf{Software Testing and Quality Assurance} and  \textbf{Requirements Analysis and System Design} at \textbf{Cranfield University}. It describes the implementation of a computing control system. This document is primarily intended to be proposed to the IT department for their approval and serve as a reference for the development of the system.


\section{Intended Audience and Reading Suggestions}
$<$Describe the different types of reader that the document is intended for, 
such as developers, project managers, marketing staff, users, testers, and 
documentation writers. Describe what the rest of this SRS contains and how it is 
organized. Suggest a sequence for reading the document, beginning with the 
overview sections and proceeding through the sections that are most pertinent to 
each reader type.$>$

\section{Project Scope}

\par The software is a \textbf{simulator of a job control system}. It will be used by the IT department of Cranfield University so that it can explore different strategies to their current implementation. 
\par The developed software will include an \textbf{User Friendly Interface}, so that it can be used more easily. The simulation shall be capable of regulate its \textbf{inputs} so that it can compute a set of outputs.
\par The resulting application shall be a \textbf{reliable} and \textbf{efficient}, \textbf{cross-platform} program.

\section{References}
$<$List any other documents or Web addresses to which this SRS refers. These may 
include user interface style guides, contracts, standards, system requirements 
specifications, use case documents, or a vision and scope document. Provide 
enough information so that the reader could access a copy of each reference, 
including title, author, version number, date, and source or location.$>$


\chapter{Overall Description}

\section{Product Perspective}
\par The product is a stand-alone system. This system shall contain \textbf{at least} 128 nodes with \textbf{at least} 16 cores per node. It will be used by a set of simulated users that can be classified as:

\begin{itemize}
\item IT support
\item Researchers
\item Students
\end{itemize}

\par The IT support simulated users have an \textbf{infinite} budget, therefore is permitted to them to run as many jobs as they like. In other hand, the Students shall have a constant budget, which is a smaller amount compared to the Researchers budget. This budget confines the amount of jobs that a user is permitted to run. The system usage has a price per core, that shall be decreased from the simulated budget every second.
\par The users can use the system by submitting jobs. This jobs have a two main characteristics, the amount of time that will use the system (running time), and the amount of system cores it will use. Thus, there are four types of jobs:

\begin{itemize}
\item Short - can take up to 2 nodes for no more than 1 hour. 10\% of the machine is reserved for these kind of jobs.
\item Medium - can take up for 10\% of the total number of cores for no more than 8 hours. 30\% of the system is reserved for this queue.
\item Large - can take up for 50\% of the total number of cores for no more than 16 hours. 70\% of the system is reserved for this queue.
\item Huge - can only run from 1700 of Friday to 0900 of Monday, reserving the whole machine. During this time, no other job can be executed.
\end{itemize}

\par Every time a simulated user submits a job, a scheduler shall define whenever that job is going to run. This scheduler manages the amount of computational resources at every second.


\section{Product Functions}
An user of the simulation shall be able to regulate a set of the inputs:

\begin{itemize}
\item Number of jobs \footnotemark
\item Number of users \footnotemark[\value{footnote}]
\item Student Budget \footnotemark[\value{footnote}]
\item Researcher Budget \footnotemark[\value{footnote}]
\item Simulation date - the date when the first job submission is done.
\item Requests span - the time span that a job can be submitted in.
\item Number of nodes
\item Number of cores
\end{itemize}

\footnotetext{The user shall decide whether this number is randomly defined following a linear distribution or is a constant value.}


\section{Operating Environment}
$<$Describe the environment in which the software will operate, including the 
hardware platform, operating system and versions, and any other software 
components or applications with which it must peacefully coexist.$>$

\section{Design and Implementation Constraints}
$<$Describe any items or issues that will limit the options available to the 
developers. These might include: corporate or regulatory policies; hardware 
limitations (timing requirements, memory requirements); interfaces to other 
applications; specific technologies, tools, and databases to be used; parallel 
operations; language requirements; communications protocols; security 
considerations; design conventions or programming standards (for example, if the 
customer’s organization will be responsible for maintaining the delivered 
software).$>$

\section{User Documentation}
$<$List the user documentation components (such as user manuals, on-line help, 
and tutorials) that will be delivered along with the software. Identify any 
known user documentation delivery formats or standards.$>$
\section{Assumptions and Dependencies}

$<$List any assumed factors (as opposed to known facts) that could affect the 
requirements stated in the SRS. These could include third-party or commercial 
components that you plan to use, issues around the development or operating 
environment, or constraints. The project could be affected if these assumptions 
are incorrect, are not shared, or change. Also identify any dependencies the 
project has on external factors, such as software components that you intend to 
reuse from another project, unless they are already documented elsewhere (for 
example, in the vision and scope document or the project plan).$>$


\chapter{External Interface Requirements}

\section{User Interfaces}
$<$Describe the logical characteristics of each interface between the software 
product and the users. This may include sample screen images, any GUI standards 
or product family style guides that are to be followed, screen layout 
constraints, standard buttons and functions (e.g., help) that will appear on 
every screen, keyboard shortcuts, error message display standards, and so on.  
Define the software components for which a user interface is needed. Details of 
the user interface design should be documented in a separate user interface 
specification.$>$

\section{Hardware Interfaces}
$<$Describe the logical and physical characteristics of each interface between 
the software product and the hardware components of the system. This may include 
the supported device types, the nature of the data and control interactions 
between the software and the hardware, and communication protocols to be 
used.$>$

\section{Software Interfaces}
$<$Describe the connections between this product and other specific software 
components (name and version), including databases, operating systems, tools, 
libraries, and integrated commercial components. Identify the data items or 
messages coming into the system and going out and describe the purpose of each.  
Describe the services needed and the nature of communications. Refer to 
documents that describe detailed application programming interface protocols.  
Identify data that will be shared across software components. If the data 
sharing mechanism must be implemented in a specific way (for example, use of a 
global data area in a multitasking operating system), specify this as an 
implementation constraint.$>$

\section{Communications Interfaces}
$<$Describe the requirements associated with any communications functions 
required by this product, including e-mail, web browser, network server 
communications protocols, electronic forms, and so on. Define any pertinent 
message formatting. Identify any communication standards that will be used, such 
as FTP or HTTP. Specify any communication security or encryption issues, data 
transfer rates, and synchronization mechanisms.$>$


\chapter{System Features}
$<$This template illustrates organizing the functional requirements for the 
product by system features, the major services provided by the product. You may 
prefer to organize this section by use case, mode of operation, user class, 
object class, functional hierarchy, or combinations of these, whatever makes the 
most logical sense for your product.$>$

\section{System Feature 1}
$<$Don’t really say “System Feature 1.” State the feature name in just a few 
words.$>$

\subsection{Description and Priority}
$<$Provide a short description of the feature and indicate whether it is of 
High, Medium, or Low priority. You could also include specific priority 
component ratings, such as benefit, penalty, cost, and risk (each rated on a 
relative scale from a low of 1 to a high of 9).$>$

\subsection{Stimulus/Response Sequences}
$<$List the sequences of user actions and system responses that stimulate the 
behavior defined for this feature. These will correspond to the dialog elements 
associated with use cases.$>$

\subsection{Functional Requirements}
$<$Itemize the detailed functional requirements associated with this feature.  
These are the software capabilities that must be present in order for the user 
to carry out the services provided by the feature, or to execute the use case.  
Include how the product should respond to anticipated error conditions or 
invalid inputs. Requirements should be concise, complete, unambiguous, 
verifiable, and necessary. Use “TBD” as a placeholder to indicate when necessary 
information is not yet available.$>$

$<$Each requirement should be uniquely identified with a sequence number or a 
meaningful tag of some kind.$>$

REQ-1:	REQ-2:

\section{System Feature 2 (and so on)}


\chapter{Other Nonfunctional Requirements}

\section{Performance Requirements}
$<$If there are performance requirements for the product under various 
circumstances, state them here and explain their rationale, to help the 
developers understand the intent and make suitable design choices. Specify the 
timing relationships for real time systems. Make such requirements as specific 
as possible. You may need to state performance requirements for individual 
functional requirements or features.$>$

\section{Safety Requirements}
$<$Specify those requirements that are concerned with possible loss, damage, or 
harm that could result from the use of the product. Define any safeguards or 
actions that must be taken, as well as actions that must be prevented. Refer to 
any external policies or regulations that state safety issues that affect the 
product’s design or use. Define any safety certifications that must be 
satisfied.$>$

\section{Security Requirements}
$<$Specify any requirements regarding security or privacy issues surrounding use 
of the product or protection of the data used or created by the product. Define 
any user identity authentication requirements. Refer to any external policies or 
regulations containing security issues that affect the product. Define any 
security or privacy certifications that must be satisfied.$>$

\section{Software Quality Attributes}
$<$Specify any additional quality characteristics for the product that will be 
important to either the customers or the developers. Some to consider are: 
adaptability, availability, correctness, flexibility, interoperability, 
maintainability, portability, reliability, reusability, robustness, testability, 
and usability. Write these to be specific, quantitative, and verifiable when 
possible. At the least, clarify the relative preferences for various attributes, 
such as ease of use over ease of learning.$>$

\section{Business Rules}
$<$List any operating principles about the product, such as which individuals or 
roles can perform which functions under specific circumstances. These are not 
functional requirements in themselves, but they may imply certain functional 
requirements to enforce the rules.$>$


\chapter{Other Requirements}
$<$Define any other requirements not covered elsewhere in the SRS. This might 
include database requirements, internationalization requirements, legal 
requirements, reuse objectives for the project, and so on. Add any new sections 
that are pertinent to the project.$>$

\section{Appendix A: Glossary}
%see https://en.wikibooks.org/wiki/LaTeX/Glossary
$<$Define all the terms necessary to properly interpret the SRS, including 
acronyms and abbreviations. You may wish to build a separate glossary that spans 
multiple projects or the entire organization, and just include terms specific to 
a single project in each SRS.$>$

\section{Appendix B: Analysis Models}
$<$Optionally, include any pertinent analysis models, such as data flow 
diagrams, class diagrams, state-transition diagrams, or entity-relationship 
diagrams.$>$

\section{Appendix C: To Be Determined List}
$<$Collect a numbered list of the TBD (to be determined) references that remain 
in the SRS so they can be tracked to closure.$>$

\end{document}
