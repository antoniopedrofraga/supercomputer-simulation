
%%%%%%%%%%%%%%%%%%%%%%%%%%%%%%%%%%%%%%%%%%%%%%%%%%%%%%%%%%%%%%%%%%%%%
% LaTeX Template: Project Titlepage Modified (v 0.1) by rcx
%
% Original Source: http://www.howtotex.com
% Date: February 2014
% 
% This is a title page template which be used for articles & reports.
% 
% This is the modified version of the original Latex template from
% aforementioned website.
% 
%%%%%%%%%%%%%%%%%%%%%%%%%%%%%%%%%%%%%%%%%%%%%%%%%%%%%%%%%%%%%%%%%%%%%%

\documentclass[12pt]{article}
\usepackage[utf8]{inputenc}
\usepackage[a4paper]{geometry}
\usepackage[myheadings]{fullpage}
\usepackage{enumitem}
\usepackage{fancyhdr}
\usepackage{lastpage}
\usepackage{graphicx, wrapfig, subcaption, setspace, booktabs}
\usepackage[T1]{fontenc}
\usepackage[font=small, labelfont=bf]{caption}
\usepackage{fourier}
\usepackage{amsmath}
\usepackage[protrusion=true, expansion=true]{microtype}
\usepackage[english]{babel}
\usepackage{sectsty}
\usepackage{url, lipsum}
\usepackage{titlesec}
\usepackage{diagbox}
\usepackage{pdfpages}

\usepackage{listings}
\usepackage{color}

\definecolor{dkgreen}{rgb}{0,0.6,0}
\definecolor{gray}{rgb}{0.5,0.5,0.5}
\definecolor{mauve}{rgb}{0.58,0,0.82}

\lstset{frame=tb,
  language=C++,
  aboveskip=3mm,
  belowskip=3mm,
  showstringspaces=false,
  columns=flexible,
  basicstyle={\small\ttfamily},
  numbers=none,
  numberstyle=\tiny\color{gray},
  keywordstyle=\color{blue},
  commentstyle=\color{dkgreen},
  stringstyle=\color{mauve},
  breakatwhitespace=true,
  breaklines=true,
  tabsize=2
}

\newcommand{\HRule}[1]{\rule{\linewidth}{#1}}
\onehalfspacing
\setcounter{tocdepth}{5}
\setcounter{secnumdepth}{5}
\inputencoding{utf8}

\titleformat{\paragraph}
{\normalfont\normalsize\bfseries}{\theparagraph}{1em}{}
\titlespacing*{\paragraph}
{0pt}{3.25ex plus 1ex minus .2ex}{1.5ex plus .2ex}

%-------------------------------------------------------------------------------
% HEADER & FOOTER
%-------------------------------------------------------------------------------
\pagestyle{fancy}
\fancyhf{}
\setlength\headheight{15pt}
\fancyhead[L]{António Pedro Araújo Fraga}
\fancyhead[R]{Cranfield University}
\fancyfoot[R]{Page \thepage\ of \pageref{LastPage}}
%-------------------------------------------------------------------------------
% TITLE PAGE
%-------------------------------------------------------------------------------

\begin{document}

\title{ \fontsize{40}{90} \textsc{Test Plan}
		\\ [2.0cm]
		\HRule{0.5pt} \\
		\LARGE \textbf{Simulated computing system}
		\HRule{2pt} \\ [0.5cm]
		\normalsize \today \vspace*{5\baselineskip}}

\date{}

\author{
		\textbf{António Pedro Araújo Fraga} \\
		\textbf{Student ID: 279654} \\ 
		\textbf{Cranfield University} \\
		\textbf{M.Sc. in Software Engineering for Technical Computing
		} }

\maketitle
\thispagestyle{empty}
\newpage
\tableofcontents
\thispagestyle{empty}
\newpage

%-------------------------------------------------------------------------------
% Section title formatting
\sectionfont{\scshape}
\titleformat{\section}
{\normalfont\huge\bfseries}{\thesection}{1em}{}
\titleformat{\subsection}
{\normalfont\large\bfseries}{\thesubsection}{1em}{}
\titlespacing*{\section}
{0pt}{5.5ex plus 1ex minus .2ex}{4.3ex plus .2ex}
\titlespacing*{\subsection}
{0pt}{5.5ex plus 1ex minus .2ex}{4.3ex plus .2ex}
%-------------------------------------------------------------------------------

%-------------------------------------------------------------------------------
% BODY
%-------------------------------------------------------------------------------

%-------------------------------------------------------------------------------
% Introduction
%-------------------------------------------------------------------------------

\section*{Test Plan Identifier}
\addcontentsline{toc}{section}{Test Plan Identifier}

Simulated Computing System, Fraga

\section*{References}
\addcontentsline{toc}{section}{References}

This test plan is based on the \textbf{IEEE 829 format}, regarding to a test plan outline. 


\section*{Introduction}
\addcontentsline{toc}{section}{Introduction}
\par This system will be developed under two modules, \textbf{Software Testing and Quality Assurance} and  \textbf{Requirements Analysis and System Design} at \textbf{Cranfield University}. 
\par It is supposed to simulate a computer system which runs jobs of several sizes. The jobs last between \textbf{one second} and \textbf{thirty eight hours}, assigning them with a category of \textbf{Short}, \textbf{Medium}, \textbf{Large} or \textbf{Huge}. The system shall be capable of scheduling a job running time on a basis of a \textbf{First Come, First Served} methodology, analysing what is the correct time to run a job every time a submission is made. The simulation must generate a set of informations regarding the pretended outputs. These outputs are dependent of inputs values that will be established in the beginning of the simulation.


\par The application is going to be used by the IT department, modelling the behaviour of a real computing platform and exploring alternative accounting strategies.

\par The system shall attend functional and non-functional requirements. Most of functional requirements shall be tested at the \textbf{Unit \& Integration level}, whereas non-functional requirements shall be tested at a higher level.

\section*{Features and Functions to test}
\addcontentsline{toc}{section}{Features and Functions to test}

\begin{itemize}  
\item Coding standards
\item Compatibility 
\item Functional 
\item Reliability
\end{itemize}

\section*{Approach/Strategy}
\addcontentsline{toc}{section}{Approach/Strategy}

\subsection*{Coding standards}
\addcontentsline{toc}{subsection}{Coding standards}

\par The developed code shall follow the coding standards written in the \textbf{Software Requirements Specification} document. In order to achieve this, it is going to be used a checker as an IDE plug in, called \textbf{cppchecker}.

\subsection*{Compatibility}
\addcontentsline{toc}{subsection}{Compatibility}

\par The system must be capable of running in different environments, therefore, the system functional requirements shall be tested in different machines, running different \textbf{Operating Systems}.


\subsection*{Functional}
\addcontentsline{toc}{subsection}{Functional}

\par Functional requirements must be tested with Unit Tests, making use of a proper tool (\textbf{Catch}). These requirements must also be tested on the \textbf{version control} system, making use of \textbf{Travis}. This implementation will allow \textbf{integration testing}.

\subsection*{Reliability}
\addcontentsline{toc}{subsection}{Reliability}

\par The system shall not crash when put under \textbf{Stress Tests}. Extreme situations must be tested, expecting correct functioning and results.

\section*{Item Pass/Fail Criteria}
\addcontentsline{toc}{section}{Item Pass/Fail Criteria}

\par Every feature must be tested and approved.

\end{document}

